\documentclass[12pt]{article}

\usepackage{cmap}
\usepackage[T2A]{fontenc}
\usepackage[utf8]{inputenc}
\usepackage[russian]{babel}
\usepackage{graphicx}
\usepackage{amsthm,amsmath,amssymb}
\usepackage[russian,colorlinks=true,urlcolor=red,linkcolor=blue]{hyperref}
\usepackage{enumerate}
\usepackage{datetime}
\usepackage{minted}
\usepackage{fancyhdr}
\usepackage{lastpage}
\usepackage{color}
\usepackage{verbatim}
\usepackage{tikz}
\usepackage{epstopdf}
\usepackage{xifthen}

\parskip=0em
\parindent=0em

\sloppy
\voffset=-20mm
\textheight=235mm
\hoffset=-25mm
\textwidth=180mm
\headsep=12pt
\footskip=20pt

\setcounter{page}{0}
\pagestyle{empty}

% Основные математические символы
\DeclareSymbolFont{extraup}{U}{zavm}{m}{n}
\DeclareMathSymbol{\heart}{\mathalpha}{extraup}{86}
\newcommand{\N}{\mathbb{N}}   % Natural numbers
\newcommand{\R}{\mathbb{R}}   % Ratio numbers
\newcommand{\Z}{\mathbb{Z}}   % Integer numbers
\def\EPS{\varepsilon}         %
\def\SO{\Rightarrow}          % =>
\def\EQ{\Leftrightarrow}      % <=>
\def\t{\texttt}               % mono font
\def\c#1{{\rm\sc{#1}}}        % font for classes NP, SAT, etc
\def\O{\mathcal{O}}           %
\def\NO{\t{\#}}               % #
\renewcommand{\le}{\leqslant} % <=, beauty
\renewcommand{\ge}{\geqslant} % >=, beauty
\def\XOR{\text{ {\raisebox{-2pt}{\ensuremath{\Hat{}}}} }}
\newcommand{\q}[1]{\langle #1 \rangle}               % <x>
\newcommand\URL[1]{{\footnotesize{\url{#1}}}}        %
\newcommand{\sfrac}[2]{{\scriptstyle\frac{#1}{#2}}}  % Очень маленькая дробь
\newcommand{\mfrac}[2]{{\textstyle\frac{#1}{#2}}}    % Небольшая дробь
\newcommand{\score}[1]{{\bf\color{red}{(#1)}}}

% Отступы
\def\makeparindent{\hspace*{\parindent}}
\def\up{\vspace*{-0.3em}}
\def\down{\vspace*{0.3em}}
\def\LINE{\vspace*{-1em}\noindent \underline{\hbox to 1\textwidth{{ } \hfil{ } \hfil{ } }}}
%\def\up{\vspace*{-\baselineskip}}

\renewcommand{\headrulewidth}{0.4pt}

\lfoot{}
\cfoot{\thepage\t{/}\pageref*{LastPage}}
\rfoot{}
\renewcommand{\footrulewidth}{0.4pt}

\newenvironment{MyList}[1][4pt]{
  \begin{enumerate}[1.]
  \setlength{\parskip}{0pt}
  \setlength{\itemsep}{#1}
}{       
  \end{enumerate}
}
\newenvironment{InnerMyList}[1][0pt]{
  \vspace*{-0.5em}
  \begin{enumerate}[a)]
  \setlength{\parskip}{#1}
  \setlength{\itemsep}{0pt}
}{
  \end{enumerate}
}

\newcommand{\Section}[1]{
  \refstepcounter{section}
  \addcontentsline{toc}{section}{\arabic{section}. #1} 
  %{\LARGE \bf \arabic{section}. #1} 
  {\LARGE \bf #1} 
  \vspace*{1em}
  \makeparindent\unskip
}
\newcommand{\Subsection}[1]{
  \refstepcounter{subsection}
  \addcontentsline{toc}{subsection}{\arabic{section}.\arabic{subsection}. #1} 
  {\Large \bf \arabic{section}.\arabic{subsection}. #1} 
  \vspace*{0.5em}
  \makeparindent\unskip
}

% Код с правильными отступами
\newenvironment{code}{
  \VerbatimEnvironment

  \vspace*{-0.5em}
  \begin{minted}{c}}{
  \end{minted}
  \vspace*{-0.5em}

}

% Формулы с правильными отступами
\newenvironment{smallformula}{
 
  \vspace*{-0.8em}
}{
  \vspace*{-1.2em}
  
}
\newenvironment{formula}{
 
  \vspace*{-0.4em}
}{
  \vspace*{-0.6em}
  
}


\definecolor{dkgreen}{rgb}{0,0.6,0}
\definecolor{brown}{rgb}{0.5,0.5,0}
\newcommand{\red}[1]{{\color{red}{#1}}}
\newcommand{\dkgreen}[1]{{\color{dkgreen}{#1}}}

\begin{document}

\begin{MyList}
	\item[2.]
	Избавимся от комбинированных и длинных правил:
	\begin{MyList}
		\item $S \to RS \ | \ R$
		\item $R \to AT_1 \ | \ CT_2 \ | \ AB \ | \ CD \ | \ \varepsilon$
		\item $T_1 \to SB$
		\item $T_2 \to RD$
		\item $A \to a$
		\item $B \to b$
		\item $C \to c$
		\item $D \to d$
	\end{MyList}
	
	Удалим $\varepsilon$-правила.
	\begin{MyList}
		\item $S \to RS \ | \ R \ | \ \varepsilon$
		\item $R \to AT_1 \ | \ CT_2 \ | \ AB \ | \ CD $
		\item $T_1 \to SB$
		\item $T_2 \to RD \ | D$
		\item $A \to a$
		\item $B \to b$
		\item $C \to c$
		\item $D \to d$
	\end{MyList}
	
	Поменяем стартовое состояние:
	\begin{MyList}
		\item $S' \to S \ | \ \varepsilon$
		\item $S \to RS \ | \ R$
		\item $R \to AT_1 \ | \ CT_2 \ | \ AB \ | \ CD $
		\item $T_1 \to SB$
		\item $T_2 \to RD \ | D$
		\item $A \to a$
		\item $B \to b$
		\item $C \to c$
		\item $D \to d$
	\end{MyList}
	
	Удалим унарные правила:
	\begin{MyList}
		\item $S' \to S \ | \ \varepsilon$
		\item $S \to RS \ | \ AT_1 \ | \ CT_2 \ | \ AB \ | \ CD $
		\item $R \to AT_1 \ | \ CT_2 \ | \ AB \ | \ CD $
		\item $T_1 \to SB$
		\item $T_2 \to RD \ | \ d$
		\item $A \to a$
		\item $B \to b$
		\item $C \to c$
		\item $D \to d$
	\end{MyList}
	
	\item[3.] Является.
	
	Грамматика: $S \to aaS\ |\ Sbb\ |\ aSb\ |\ ab\ |\ aa\ |\ bb$
	
	Так как каждый раз мы добавляем по 2 символа, добавляя $a$ только в начало и $b$ только в конец, то в итоге получится что-то из $a^mb^n: (m + n)$ кратно 2\\
	Теперь осталось показать что для любых $w = a^mb^n:$ таких что $(m + n)$ кратно 2 существует какой-то вывод\\
	Применим правило $S \to aSb$ $min(m,n)$ раз, если $m = n$ то на последнем шаге вместо  $S \to aSb$ применим $S \to ab$.\\ 
	Осталось добавить еще $(m - n)$ букв $a$ или $(n - m)$ букв $b$ (если $m \ne n$).\\
	Т.к $m + n = m + m + (n - m) = n + n + (m - n)$ четное то $m - n$ и $n - m$ тоже четные, а значит правилами $aaS$ или $Sbb$ мы можем добавить сколько надо, но на последнем шаге вместо $aaS$ нужно использовать $aa$, а вместо $Sbb$ $bb$\\
\end{MyList}

\end{document}
